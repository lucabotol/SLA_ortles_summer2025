\section{Inquadramento dell'area di studio}
L'area di studio di questo lavoro riguarda il versante trentino del gruppo montuoso dell'Ortles-Cevedale, nella Comunità della Val di Sole, al confine con la Regione Lombardia (\cref{fig:pat_valdisole}, \ref{fig:dtm_valdisole}, \ref{fig:area_interesse}).\\
Il gruppo raggiunge la sua massima elevazione con la Cima dell'Ortles, che tocca i 3905 m s.l.m.. L'area è caratterizzata da una morfologia alpina complessa, con valli laterali significative come la Val di Trafoi, la Val Martello e la Val d'Ultimo \cite{crippa_2025}.\\
Il clima dell'area ha un carattere prevalentemente continentale. Le fonti evidenziano una variabilità regionale nelle precipitazioni: il settore occidentale (che include la Val Venosta e le sue valli laterali come la Val di Trafoi e la Val Martello) riceve mediamente meno precipitazioni (inferiori o uguali a 825,2 mm annui) rispetto ai settori centrali e orientali delle Alpi sud-orientali \cite{seppi_2011}.\\
L'area dell'Ortles-Cevedale è una regione cruciale per lo studio del permafrost e dei ghiacciai, sia per comprendere il loro ruolo nel ciclo idrogeologico e sia come indicatori del cambiamento climatico globale \cite{seppi_2011} \cite{andreetta_2024}.

\begin{figure}[hbt]
    \centering
    \includegraphics[width=0.8\textwidth]{immagini/pat_valdisole.pdf}
    \caption{Inquadramento geografico dell'area di studio. In cartografia è indicato il territorio della Provincia Autonoma di Trento, con il particolare riferimento alla Val di Sole (PAT e Google Satellite).}
    \label{fig:pat_valdisole}
\end{figure}
\begin{figure}[hbt]
    \centering
    \includegraphics[width=0.8\textwidth]{immagini/dtm_valdisole.pdf}
    \caption{Rappresentazione mediante DTM e hillshade della Val di Sole (TN).}
    \label{fig:dtm_valdisole}
\end{figure}
\begin{figure}[hbt]
    \centering
    \includegraphics[width=0.8\textwidth]{immagini/area_interesse.pdf}
    \caption{Zona d'interesse considerata ai fini dello studio; ovvero l'area più a nord-ovest della Val di Sole (TN).}
    \label{fig:area_interesse}
\end{figure}

