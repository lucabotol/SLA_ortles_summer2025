\section{Inquadramento dell'area di studio}
I rilievi sono stati svolti all'interno della Foresta del Cansiglio, più precisamente in zona Valmenera, nel Comune di Tambre (BL) (\cref{fig:inq_cansiglio} \cref{fig:inq_cansiglio_2}).
\begin{figure}[H]
    \centering
    \includegraphics[width=0.8\textwidth]{immagini/inq_cansiglio.pdf}
    \caption{Inquadramento geografico dell'area di studio. In cartografia è indicato il territorio del Cansiglio (Google Satellite). I punti rossi indicano le aree di rilievo in condizioni non disturbate da schianti da vento, mentre i punti verdi indicano i rilievi in aree soggette a VAIA.}
   \label{fig:inq_cansiglio}
\end{figure}

\begin{figure}[H]
    \centering
    \includegraphics[width=0.8\textwidth]{immagini/inq_cansiglio_2.pdf}
    \caption{Inquadramento geografico dell'area di studio, in condizioni pre-evento VAIA (ESRI Satellite). I punti rossi indicano le aree di rilievo in condizioni non disturbate da schianti da vento, mentre i punti verdi indicano i rilievi in aree soggette a VAIA.}    \label{fig:inq_cansiglio_2}
\end{figure}
%\begin{figure}[H]
%    \centering
%    \includegraphics[width=0.7\textwidth]{}
%    \caption{Inquadramento geografico dell'area di studio. In cartografia è indicato il territorio della Provincia Autonoma di Trento, con il particolare riferimento alla Val di Sole e Val de la Mare (PAT e Google Satellite).}
%    \label{fig:pat_valdisole}
%\end{figure}
La pluviometria del Cansiglio, analizzando le precipitazioni mensili nel periodo 1994-2021, si attesta ad un valore di 2080 mm/anno. Il valore è superiore rispetto alla media regionale veneta, pari a 1136 mm/anno.\\
Il regime pluviometrico presenta dei massimi di pioggia in primavera e nella parte centrale dell'autunno.\\
La temperatura media annua della zona è pari a $6.1^{\circ}C$.\\
Unendo il parametro di temperatura e pluviometria, ed utilizzando la Classificazione dei climi di K\"oppen, il Cansiglio presenta un clima temperato umido con estate tiepida.\\
Data la variabilità delle condizioni climatiche dell'area del Cansiglio, si riescono ad identificare diversi tipi forestali.\\
Nella zona circostanti il Pian Cansiglio, dove il clima è mite, si trovano faggete. Abbassandosi di quota il faggio si mescola all'abete rosso e bianco, creando boschi misti molto estesi. Inoltre, si identificano boschi di abete rosso, in gran parte di origine artificiale, e grandi aree dedite al pascolo.\\
Per quanto riguarda l'aree di studio, le categorie forestali sono considerate faggete, peccete e abietete (\cref{fig:inq_cat_fores}).\\
\begin{figure}[H]
    \centering
    \includegraphics[width=0.8\textwidth]{immagini/inq_cat_fores.pdf}
    \caption{Inquadramento delle categorie forestali, con particolare nelle aree dove sono stati svolti i rilievi (Del Favero, 2006).}    \label{fig:inq_cat_fores}
\end{figure}
Dal punto di vista selvicolturale, a partire dagli anni '30, nella Foresta del Cansiglio è stato applicato il trattamento a tagli successivi, e da allora è stato ritenuto il più idoneo per gestire quella data formazione forestale, in quanto di facile applicazione e sicuro per salvaguardare la struttura monostratificata e la comparsa della rinnovazione secondo i tempi previsti.