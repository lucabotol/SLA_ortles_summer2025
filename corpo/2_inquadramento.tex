\section{Inquadramento dell'area di studio}
L'area di studio di questo lavoro riguarda il bacino della Val de la Mare, nella Comunità della Val di Sole, in Provincia Autonoma di Trento (\cref{fig:pat_valdisole}, \ref{fig:dtm_valdisole}, \ref{fig:area_interesse}).\\
Più precisamente, l'area dell'analisi comprende Cima Cavaion (3120 m s.l.m.) ed il Lago del Careser (2603 m s.l.m.), e si pone all’interno del Gruppo Ortles-Cevedale.\\
%Il clima dell'area ha un carattere prevalentemente continentale. Le fonti evidenziano una variabilità regionale nelle precipitazioni: il settore occidentale (che include la Val Venosta e le sue valli laterali come la Val di Trafoi e la Val Martello) riceve mediamente meno precipitazioni (inferiori o uguali a 825,2 mm annui) rispetto ai settori centrali e orientali delle Alpi sud-orientali \cite{seppi_2011}.\\
La temperatura, al variare della quota, possiede un gradiente di 5.4 $^{\circ}C$/1000m. È stato valutato che la quota dell’isoterma $0^{\circ}$C si trovi a 2500 m.\\
Le condizioni meteorologiche locali dell’area di studio sono monitorate da due impianti meteorologici, uno collocato presso la diga del lago Careser (2605 m s.l.m.) ed uno installato recentemente nell’area di studio del Cavaion (2891 m s.l.m.) \cite{carturan_2010}.\\
La Val de la Mare, come per l'area dell'Ortles-Cevedale, sono delle regioni cruciali per lo studio del permafrost e dei ghiacciai, sia per comprendere il loro ruolo nel ciclo idrogeologico e sia come indicatori del cambiamento climatico globale \cite{seppi_2011} \cite{andreetta_2024}.

\begin{figure}[H]
    \centering
    \includegraphics[width=0.7\textwidth]{immagini/pat_valdisole.pdf}
    \caption{Inquadramento geografico dell'area di studio. In cartografia è indicato il territorio della Provincia Autonoma di Trento, con il particolare riferimento alla Val di Sole e Val de la Mare (PAT e Google Satellite).}
    \label{fig:pat_valdisole}
\end{figure}
\begin{figure}[H]
    \centering
    \includegraphics[width=0.7\textwidth]{immagini/dtm_valdisole.pdf}
    \caption{Rappresentazione mediante DTM e hillshade della Val de la Mare (TN).}
    \label{fig:dtm_valdisole}
\end{figure}
\begin{figure}[H]
    \centering
    \includegraphics[width=0.7\textwidth]{immagini/area_interesse.pdf}
    \caption{Zona d'interesse considerata ai fini dello studio: l'alta Val de la Mare (TN).}
    \label{fig:area_interesse}
\end{figure}

Utilizzando i dati meteorologici, ricavati dalla stazione posta su Lago del Careser e successivamente corretti mediante post-processing, è possibile conoscere la climatologia della Val de la Mare, per gli anni idrologici dal 1940 al 2024 \cite{Carturan_DallaFontana_Borga_2012} (\cref{fig:clima_valpeio}).
\begin{figure}[H]
    \centering
    \includegraphics[width=0.8\textwidth]{immagini/pluviometria_temperatura_anno_idrologico.pdf}
    \caption{Precipitazione cumulata e temperatura media della Val de la Mare, dal 1940 al 2024 (anno idrologico)\cite{Carturan_DallaFontana_Borga_2012}.}
    \label{fig:clima_valpeio}
\end{figure}
È possibile notare come, soprattutto dagli anni '60, ci sia stato un chiaro aumento della temperatura media annua dell'aria; invece, per quanto riguarda la pluviometria, non sono evidenti altrettanto grandi variazioni.\\
Pur non indicando la situazione climatica per l'anno d'interesse per questa relazione (2025), il grafico permette comunque di capire quale sia stato il trend del recente passato.