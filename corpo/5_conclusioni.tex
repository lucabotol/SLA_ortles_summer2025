\section{Conclusioni}
Al fine di conoscere la condizione della rinnovazione naturale della Foresta del Cansiglio, in seguito agli schianti da vento dovuti alla Tempesta Vaia (ottobre-novembre 2018), sono state compiute delle analisi in campo: sono state analizzate 3 aree di saggio circolati di un popolamento soggetto a schianti da vento ed un transetto rettangolare comprendente una superficie intatta.\\
I risultati relativi alle aree disturbate dalla tempesta sono differenti, sia per quanto riguarda la quantità di necromassa presente e sia per la condizione di rinnovazione.\\
Il sito con la condizione di rinnovazione migliore si presentava con una copertura arborea del 90\% ed una quantità di necromassa di 181.60 $m^3/ha$. Invece, il sito con la condizione di rinnovazione peggiore presentava una copertura arborea del 5\%, con una quantità di necromassa di 116 $m^3/ha$.\\
La presenza di neve al suolo ha reso maggiormente difficoltosa la valutazione della copertura vegetale al suolo, e di conseguenza maggiormente soggetta ad errori.\\
Data l'importanza del processo di rinnovazione, risulta necessaria la ripetizione in futuro di analisi in campo.

\includepdf[pages=-,landscape=true]{immagini/kramer_doppiepagine.pdf}