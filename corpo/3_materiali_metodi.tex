\section{Materiali e metodi}
\subsection{Immagini satellitari}
Lo svolgimento di questa relazione si è basata sull'analisi delle immagini satellitari ricavate dalla missione europea di Sentinel-2 \cite{esa_sentinel2}.\\
La coppia di satelliti (con un tempo di ritorno di 10 giorno ciascuno) permette di ottenere una rappresentazione della condizione del terreno ogni 5 giorni.\\
L'acquisizione delle immagini avviene in 13 bande, con risoluzioni spaziali che possono essere di 10, 20 o 60 metri (a seconda della frequenza rilevata).\\
Le risorse acquisite dalla missione Sentinel-2 sono liberamente consultabili dal sito web del programma europeo Copernicus \cite{copernicus_browser_2026}.\\
In tutto, sono stati scaricati 17 rilievi satellitari riguardanti la zona di studio, in un periodo che intercorre tra inizio maggio a fine settembre 2025.\\
Sfortunatamente, alcune immagini satellitari sono state scartate, data la grande quantità di nuvole rilevate dal sensore.\\
Sono stati scelti, e quindi utilizzati, 13 rilievi satellitari (tabella \ref{tab:imm_sentinel}).
\begin{table}[H] \centering
    \caption{Periodi di acquisizione di dati territoriali, mediante Sentinel-2.}
\begin{tabular}{cc}
\toprule
 \makecell {Data di acquisizione \\dell'immagine}        & Annotazioni \\
\midrule
02/05/2025                            &             \\
30/05/2025                            &             \\
09/06/2025                            &             \\
13/06/2025                            &             \\
19/06/2025 &presenza di poche nuvole a bassa quota  \\
09/07/2025 &presenza di poche nuvole a bassa quota  \\
08/08/2025 &presenza di poche nuvole a bassa quota  \\
13/08/2025                            &             \\
18/08/2025                            &             \\
25/08/2025                            &             \\
17/09/2025                            &             \\
19/09/2025                            &             \\
29/09/2025      & grande presenza di neve fresca           \\
\bottomrule
\end{tabular}
\label{tab:imm_sentinel}
\end{table}

%\subsection{Perimetrazione del ghiacciaio}
%Il comportamento della neve al suolo e dei ghiacciai è correlato l'uno con l'altro, sia per cause climatico-meteorologiche che per cause topografiche.\\
%Per questo motivo, si è scelto di considerare anche il ritiro del ghiacciaio, dell'Ortles-Cevedale, confrontando la situazione al 2015 ed al 2022 (\cref{fig:ghiacciaio_2015_2022}).\\
%Osservando visivamente le due condizioni, soprattutto nella zona più settentrionale ed a quote inferiori, è possibile notare una serie di riduzioni areali del ghiacciaio.
%\begin{figure}[h]
%    \centering
%    \includegraphics[width=0.8\textwidth]{immagini/ghiacciaio_2015_2022.pdf}
%    \caption{Perimetrazione dell'estensione del ghiacciaio dell'Ortles-Cevedale al 2015 (poligoni verdi) ed al 2022 (campitura arancione).}
%    \label{fig:ghiacciaio_2015_2022}
%\end{figure}

\subsection{Indice NDSI}
Al fine della valutazione della presenza al terreno di copertura nevosa, è stato scelto di applicare alle immagini satellitari l'indice NDSI (Normalized Difference Snow Index).\\
Questo indice utilizza le bande del verde e dello SWIR , in modo da riconoscere la presenza di neve al suolo, differenziandola dalle nuvole (con riflettanza simile) \cite{NDSI_1994}: 
\begin{equation}
    NDSI = \frac{VERDE - SWIR}{VERDE + SWIR}
\end{equation} 
Utilizzando il materiale satellitare di Sentinel-2, la formula dell'indice diventa:
\begin{equation}
    NDSI = \frac{B3 - B11}{B3 + B11}
\end{equation}
Si è scelto di considerare una cella come innevata se il risultato dell'indice supera il valore soglia di 0.42 \cite{sentinel_hub_ndsi}.\\
In figura \ref{fig:ndsi_0906_1808} è possibile osservare l'applicazione dell'indice per due periodi differenti, ovvero ad inizio ed a metà estate.
\begin{figure}[H]
    \centering
    \includegraphics[width=0.7\textwidth]{immagini/ndsi_0906_1808.pdf}
    \caption{Condizioni di neve al suolo per la medesima area, secondo l'indice NDSI, al 9 giugno 2025 (a sinistra) ed al 18 agosto 2025 (a destra). Il poligono rosso indica la delimitazione della Val de la Mare. Sullo sfondo è riportata l'immagine Google Satellite.}
    \label{fig:ndsi_0906_1808}
\end{figure}


\subsection{Modello Digitale del Terreno}
Al fine di conoscere la distribuzione spaziale della neve al suolo, è stato utilizzato il modello digitale del terreno (DTM) della Val di Sole (al 2006), ritagliato sui confini della Val de la Mare. La risoluzione spaziale del DTM è di 30 m.\\
Da tale modello, mediante una maschera, sono state rimosse tutte le celle del raster con pendenza superiore $30^{\circ}$. Infatti, ad elevate pendenze, lo scivolamento verso valle della neve diventa eccessivo, inducendo degli errori nella valutazione della quota nivale di un versante (\cref{fig:dtm_valdelamare_min30gr}).\\
\begin{figure}[H]
    \centering
    \includegraphics[width=0.7\textwidth]{immagini/dtm_valdelamare_min30gr.pdf}
    \caption{DTM della Val de la Mare, depurato dalle celle con pendenza superiore i $30^{\circ}$.}
    \label{fig:dtm_valdelamare_min30gr}
\end{figure}
Infine, da ogni cella del modello risultante, sono stati estratti i centroidi (ovvero è stato creato un layer vettoriale di punti).\\
Tali punti al suolo sono stati utilizzati per campionare le informazioni di presenza neve, ricavati dall'applicazione dell'indice NDSI.

\subsection{Workflow su QGIS ed Excel}
La prima parte dello studio è stata svolta su QGIS, con il fine di conoscere la quantità di celle del raster considerate innevate (\cref{fig:model_ndsi}). \\
É stato necessario valutare visivamente la bontà delle immagini telerilevate, al fine di valutare la sufficiente assenza di copertura nuvolosa.\\
Inoltre, oltre al campionamento del raster risultante dall'indice NDSI, è stato effettuato il conteggio delle celle considerate (dal raster SCL) nuvolose. In questo modo è possibile effettuare una veloce valutazione sulla bontà del risultato dell'indice.\\
Il layer vettoriale riportante i dati campionati (delle celle innevate, nuvolose ed altimetriche) è stato importato in Excel.\\
Dopo aver suddiviso l'area in fasce altimetriche con intervalli di 20 metri, si è proceduto al conteggio dei punti ricadenti in ciascuna classe di quota ed è stato effettuato il conteggio delle celle considerate innevate e nuvolose.\\
Infine, per ogni classe altimetrica e periodo di osservazione, è stata calcolata la frazione di celle innevate rispetto al totale. Le fasce caratterizzate da una copertura nevosa superiore al $50\%$ sono state classificate come ``fasce innevate".
\begin{figure} \centering
\begin{tikzpicture}
    % 1. Definiamo i nodi (i blocchi)
    % [stile] (nome_identificativo) {testo visualizzato};
    \node[draw, rectangle] (B3) at (4,0) {B3};
    \node[draw, rectangle] (B11) at (6,0) {B11};
    \node[draw, rectangle] (area_analisi) at (9,0) {area di analisi};
    \node[draw, rectangle] (campionamento) at (4,4) {campionamento};
    \node[draw, rectangle, minimum width=3cm, align=center] (raster_calculator) at (8, 2) {calcolatore\\raster};
    \node[draw, circle] (NDSI) at (8, 4) {NDSI};
    \node[draw, rectangle] (DTM) at (13,-1.5) {DTM};
    \node[draw, rectangle] (layer_style) at (12,4) {stile del layer};
    \node[draw, rectangle] (NDSI042) at (6,7.5) {$NDSI > 0.42$};
    \node[draw, rectangle, minimum width=3cm, align=center] (occhio) at (13,0) {valutazione assenza\\ nuvole};
    \node[draw, rectangle, minimum width=3cm, align=center] (raster_calculator2) at (8, 6) {calcolatore\\raster};
    \node[draw, rectangle, minimum width=2cm, align=center] (dtm_30g) at (9, -1.5) {rimoz. celle\\pend.$<=30^{\circ}$\\del DTM};
    \node[draw, rectangle, minimum width=2cm, align=center] (punti_centroidi) at (3, -1.5) {trasformazione\\DTM$\rightarrow$punti};

    % 2. Colleghiamo i nodi con una freccia
    % [stile_freccia] (partenza) -- (arrivo);
    \draw[-{Stealth}] (B3) -- (raster_calculator);
    \draw[-{Stealth}] (B11) -- (raster_calculator);
    \draw[-{Stealth}] (occhio) -- (raster_calculator);
    \draw[-{Stealth}] (DTM) -- (dtm_30g);
    \draw[-{Stealth}] (dtm_30g) -- (area_analisi);
    \draw[-{Stealth}] (dtm_30g) -- (punti_centroidi);
    \draw[-{Stealth}] (area_analisi) -- (raster_calculator);
    \draw[-{Stealth}] (area_analisi) -- (raster_calculator);
    \draw[-{Stealth}] (raster_calculator) -- (NDSI);
    \draw[-{Stealth}] (layer_style) -- (NDSI);
    \draw[-{Stealth}] (NDSI) -- (raster_calculator2);
    \draw[-{Stealth}] (raster_calculator2) -- (NDSI042);
    \draw[-{Stealth}] (punti_centroidi) -- (campionamento);
    \draw[-{Stealth}] (campionamento) -- (NDSI042);
\end{tikzpicture} 
\caption{Schema concettuale utilizzato per ottenere il raster sulla presenza di copertura nevosa al suolo. Tale flusso di lavoro è stato utilizzato per creare un modello in QGIS.}
\label{fig:model_ndsi} \end{figure}
