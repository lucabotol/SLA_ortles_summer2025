\section{Rilievi, materiali e metodi}
I rilievi in bosco sono stati svolti mediante l'utilizzo di un cavalletto forestale (\cref{fig:rilievo_cavalletto}), un ipsometro ed una cordella metrica. In questo modo, è stato possibile tracciare le aree di saggio e rilevare i diametri e le altezze degli alberi (o lunghezze dei tronchi a terra).\\
Al fine dello svolgimento del campionamento sono state individuate 3 aree colpite dalla Tempesta Vaia ed un'area intatta. Essendo che i rilievi sono stati svolti assieme ad una collega del corso, in ogni area sono stati individuate due aree di saggio, una per ognuno.\\
Si è scelto di tracciare aree di saggio circolari, con raggio di 5 metri, per le zone soggette al disturbo di Vaia (\cref{fig:area_VAIA_1}, \cref{fig:area_VAIA_2}, \cref{fig:area_VAIA_3}); mentre, per le aree intatte da schianti, si è scelto di fare aree circolari di 10 metri di raggio (\cref{fig:area_NO_VAIA}).\\
In questo modo, le superfici di saggio soggette a disturbo sono risultate pari a 78,5 $m^2$, mentre per il bosco non disturbato pari a 314 $m^2$.\\
Come per le aree non colpite dall'evento Vaia, sono state fatte misurazioni della biomassa restante e della necromassa a terra. Data la notevole presenza di quest'ultima (\cref{fig:necromassa}) e la sua importanza nel contesto di rinnovazione naturale, sono state effettuate le misure per quantificarne la presenza in termini volumetrici. Non sono invece state condotte valutazioni sulla condizione qualitativa del materiale a terra.\\
Infine, per le aree schiantate è stata fatta una valutazione visiva della copertura, sia questa di lettiera, arborea o erbacea.\\
Com'è possibile notare dalle immagini, i rilievi sono stati svolti in condizione di presenza di neve al suolo. Pertanto, le valutazioni svolte sulla copertura del suolo sono state influenzate dalla presenza del manto nevoso.
\begin{figure}[H]
    \centering
    \includegraphics[width=0.6\textwidth]{immagini/photo_2_2026-02-05_11-06-14.jpg}
    \caption{Area di saggio non colpita dalla Tempesta Vaia, e quindi considerata intatta ai fini della relazione.}   
    \label{fig:area_NO_VAIA}
\end{figure}
\begin{figure}[H]
    \centering
    \includegraphics[width=0.6\textwidth]{immagini/photo_26_2026-02-05_11-06-14.jpg}
    \caption{Area di saggio (n.2), colpita dalla Tempesta Vaia}    
    \label{fig:area_VAIA_1}
\end{figure}
\begin{figure}[H]
    \centering
    \includegraphics[width=0.6\textwidth]{immagini/photo_17_2026-02-05_11-06-14.jpg}
    \caption{Area di saggio (n.4), colpita dalla Tempesta Vaia}   
    \label{fig:area_VAIA_2}
\end{figure}
\begin{figure}[H]
    \centering
    \includegraphics[width=0.6\textwidth]{immagini/photo_11_2026-02-05_11-06-14.jpg}
    \caption{Area di saggio (n.6), colpita dalla Tempesta Vaia}    
    \label{fig:area_VAIA_3}
\end{figure}
\begin{figure}[H]
    \centering
    \includegraphics[trim=50px 200px 50px 10px, clip, width=0.6\textwidth]{immagini/photo_25_2026-02-05_11-06-14.jpg}
    \caption{Rilievo, mediante cavalletto forestale, del DBH di un albero stroncato nel sito di rilievo 2.}
    \label{fig:rilievo_cavalletto}
\end{figure}
\begin{figure}[H]
    \centering
    \includegraphics[width=0.6\textwidth]{immagini/photo_20_2026-02-05_11-06-14.jpg}
    \caption{Particolare di un'area di saggio, comprendente un gran volume di necromassa (ceppaia ribaltata).}    
    \label{fig:necromassa}
\end{figure}

\subsection{Cenni di dendrometria}
La dendrometria rappresenta la disciplina fondamentale per la quantificazione delle risorse forestali, fornendo gli strumenti metodologici per la determinazione delle dimensioni dei singoli alberi e, per estensione, della massa legnosa dell'intero popolamento. Nel contesto della presente relazione, i rilievi si sono concentrati sui parametri biometrici principali necessari alla caratterizzazione strutturale ed ecologica del bosco.\\
Il parametro principale da valutare durante un rilievo forestale è il diametro dei tronchi (mediante l'utilizzo del cavalletto forestale), all'altezza di petto d'uomo, ovvero 1.30 metri da terra. Il diametro rilevato prende il nome di DBH (Diameter at Breast Height).\\
Mediante il diametro del fusto è possibile ottenere la superficie del tronco, definita ``Area basimetrica", spesso definita $g$:
\begin{equation}
    g = \frac{\pi}{4} \cdot d^2
\end{equation}
Avendo l'area basimetrica di tutti gli alberi all'interno della superficie di saggio, è possibile calcolare l'area basimetrica ad ettaro:
\begin{equation}
    G_{tot} = \sum_{i}^{n} g_i
\end{equation}
\begin{equation}
    G_{ha} = \frac{G_{tot}}{superficie}
\end{equation}
Conoscendo l'area basimetrica totale ed il numero di piante di tutta la superficie di saggio è possibile conoscere il diametro corrispondente alla pianta di area basimetrica media:
\begin{equation}
    g_{media}= \frac{G_{tot}}{n_{tot}}
\end{equation}
\begin{equation}
    Diam_{medio}= \sqrt{\frac{4 \cdot g_{media}}{\pi}}
\end{equation}
Conoscendo i diametri e le altezze degli alberi presenti all'interno di un'area di saggio è possibile mettere in relazione i due parametri, mediante la curva ipsometrica.\\
Questa rappresentazione esprime la variazione dell'altezza in funzione del diametro; solitamente la curva ipsometrica è specifica, altre volte può essere per gruppi di specie simili.\\
La funzione più utilizzata è quella logaritmica (logaritmo naturale del diametro). Considerando la funzione della curva interpolatrice, è possibile ricavare l'altezza dell'individuo conoscendo il diametro del tronco. In questo modo, dopo aver calcolato il diametro dell'individuo medio del popolamento, è possibile ricavare l'altezza media tra le piante del campione.
\subsection{Necromassa}
La necromassa è il legno che deriva dagli alberi morti
Fino a qualche tempo fa la necromassa era di scarso interesse in ambito forestale, oltre che essere poco presente nei boschi europei ed italiani.\\
Oggi la necromassa suscita un gran interesse per diversi motivi:
\begin{itemize}
    \item conservazione della biodiversità;
    \item bilancio del carbonio;
    \item studio dei disturbi naturali;
    \item supporto alla rinnovazione, mediante rilascio di nutrimenti.
\end{itemize}
La necromassa d'interesse in bosco è costituita dagli elementi di una certa dimensione:
\begin{itemize}
    \item snag, gli alberi morti in piedi;
    \item log, i tronchi a terra;
    \item stump, le ceppaie.
\end{itemize}
All'interno del bosco, la necromassa tende ad essere distribuita in modo più eterogeneo rispetto alla componente viva.\\
La quantità di necromassa dipende prevalentemente da:
\begin{itemize}
    \item regime di disturbi;
    \item produttività del sito;
    \item clima (relativamente alla degradazione);
    \item interventi antropici.
\end{itemize}
Per le foreste temperate senza disturbi ad alta severità, la quantità di necromassa è:
\begin{itemize}
    \item selvicoltura intensiva: 5-15 $m^3/ha$;
    \item selvicoltura estensiva / non gestite: 10-80 $m^3/ha$;
    \item old-growth (vetuste): 100-400 $m^3/ha$.
\end{itemize}
Il valore per garantire una biodiversità desiderabile è pari al range di 20-40 $m^3/ha$.
\subsection{Rilievi in campo}
\subsubsection{Area non colpita}
Quest'area di saggio si trova ad un'altitudine di 1050 m s.l.m., con un'esposizione a N-O ed una pendenza del 8.5\%.\\ 
%L'area di saggio circolare, con raggio di 10 metri, possiede una superficie di 314 metri quadri.\\
Le specie arboree rilevate all'interno dell'area sono: l'abete rosso (\textit{Picea abies}), l'abete bianco (\textit{Abies alba}) ed il faggio (\text{Fagus sylvatica}) (\cref{fig:area_NO_VAIA}).
\begin{table}[H] \centering
\caption{Valori dendrometrici ottenuti dalle misure svolte nella particella di bosco non soggetta a schianti da vento. FS=faggio, AA=abete bianco e PA=abete rosso.}
\begin{tabular}{ccccc}
    \toprule
Specie & DBH (cm) & classe (cm)& H (m) & Area bas. ($m^2$)\\
    \midrule
FS & 61 & 60 &  24 & 0.29\\
FS & 38 & 40 &  23 & 0.11\\
FS & 31 & 30 &  21 & 0.08\\
FS & 31 & 30 &  21 & 0.08\\
AA & 34 & 35 &  28 & 0.09\\
AA & 12 & 10 &  14 & 0.01\\
AA & 20 & 20 &  19 & 0.03\\
AA & 24 & 25 &  23 & 0.05\\
AA & 14 & 15 &  15 & 0.02\\
%PA & 90 & 90 &  40\\
PA & 19 & 20 &  19 & 0.03\\
PA & 51 & 51 &  33 & 0.20\\
PA & 82 & 80 &  38 & 0.53\\
PA & 65 & 65 &  35 & 0.33\\
    \bottomrule
\end{tabular}
\label{tab:area_novaia}
\end{table}
\subsubsection{Area colpita n.2}
Quest'area di saggio si trova ad un'altitudine di 1075 m s.l.m. Non è possibile estrarre ulteriori informazioni (pendenza ed esposizione), poichè il DEM in quest'area presenta un evidente distorsione.\\
\begin{table}[H] \centering
\caption{Valori dendrometrici di densità della particella di bosco soggetta a schianti da vento. L'altezza indica la lunghezza della ceppaia/fusto stroncata o tagliata. CT=ceppaia tagliata, CS=ceppaia stroncata (\cref{fig:area_VAIA_1}).}
\begin{tabular}{ccc}
    \toprule
Albero & Diametro (cm) & H (m) \\
    \midrule
CT & 24 & 0.90 \\
CS & 51 & 1.20 \\
SNAG & 44 & 7.50 \\
\bottomrule
\end{tabular}
\end{table}
\begin{table}[H] \centering
\caption{Valori della copertura a terra, ottenuti mediante valutazione visiva.}
\begin{tabular}{cc}
    \toprule
Tip. copertura & $\%$  \\
    \midrule
Arborea & 90\\
Erbacea & 10\\
Lettiera & 0\\
\bottomrule
\end{tabular}
\end{table}
\subsubsection{Area colpita n.4}
Quest'area di saggio si trova ad un'altitudine di 1056 m s.l.m., con un'esposizione a S-E ed una pendenza del 28\%.\\
\begin{table}[H] \centering
\caption{Valori dendrometrici di densità della particella di bosco soggetta a schianti da vento. L'altezza indica la lunghezza della ceppaia/fusto stroncata o tagliata. CT=ceppaia tagliata, CS=ceppaia stroncata, L=log (\cref{fig:area_VAIA_2}).}
\begin{tabular}{ccc}
    \toprule
Albero & Diametro (cm) & H (m) \\
    \midrule
CS & 63 & 0.60 \\
CS & 35 & 0.40 \\
L & 10 & 4.00 \\
L & 19 & 15.00 \\
CT & 30 & 0.30 \\
CT & 60 & 0.80 \\
CT & 30 & 0.40 \\
\bottomrule
\end{tabular}
\end{table}
\begin{table}[H] \centering
\caption{Valori della copertura a terra, ottenuti mediante valutazione visiva.}
\begin{tabular}{cc}
    \toprule
Tip. copertura & $\%$  \\
    \midrule
Arborea & 40\\
Erbacea & 30\\
Lettiera & 30\\
\bottomrule
\end{tabular}
\end{table}
\subsubsection{Area colpita n.6}
Quest'area di saggio si trova ad un'altitudine di 1038 m s.l.m., con un'esposizione a S-O ed una pendenza del 19\%.\\
\begin{table}[H] \centering
\caption{Valori dendrometrici di densità della particella di bosco soggetta a schianti da vento. L'altezza indica la lunghezza della ceppaia/fusto stroncata o tagliata. CT=ceppaia tagliata, CS=ceppaia stroncata, FS=\text{Fagus sylvatica} (\cref{fig:area_VAIA_3}).}
\begin{tabular}{ccc}
    \toprule
Albero & Diametro (cm) & H (m) \\
    \midrule
CS & 56 & 1.00 \\
CT & 50 & 0.40 \\
CT & 32 & 0.33 \\
CT & 64 & 0.55 \\
CT & 45 & 0.37 \\
CT & 65 & 0.55 \\
SNAG & 35 & 0.70 \\
CS & 39 & 0.62 \\
FS (cimale spezz.)& 19 & 7.50 \\
\bottomrule
\end{tabular}
\end{table}
\begin{table}[H] \centering
\caption{Valori della copertura a terra, ottenuti mediante valutazione visiva.}
\begin{tabular}{cc}
    \toprule
Tip. copertura & $\%$  \\
    \midrule
Arborea & 5\\
Erbacea & 85\\
Lettiera & 10\\
\bottomrule
\end{tabular}
\end{table}
\subsection{Utilizzo di remote sensing}
Al fine di valutare la severità della tempesta Vaia nel popolamento forestale di nostro interesse, è stato fatto uso di immagini satellitari (remote sensing).\\
L'utilizzo di tali risorse informatiche permette, in via teorica, di disporre in continuo di un gran numero di rilievi, con una elevata risoluzione spaziale e temporale.\\
Esistono diversi indici per valutare l'impatto di un disturbo in una foresta, e si differenziano principalmente per le caratteristiche vegetali che vanno ad analizzare (come per esempio la produzione di clorofilla, lo stress idrico, l'apertura della chioma, etc). Nella pratica, i diversi indici variano in base alle bande d'interesse utilizzate.\\
In questa relazione applicherò l'indice NDWI8A (Normalized Difference Water Index con la banda B8A), che viene considerato quello più efficace per l'analisi di vaste aree interessate da schianti da vento \cite{index_sentinel2}.\\
La formula dell'indice è simile a quella del NDWI, solamente con l'utilizzo della banda B8A al posto della B08:
\begin{equation}
    NDWI8A = \frac{B8A-B11}{B8A+B11}
\end{equation}
Il NDWI8A, in modo similare al NDWI, permette di identificare il contenuto idrico delle piante, potendo così valutare la presenza di danni all'organismo o eventuali esaurimenti di risorse.\\
Il risultato dell'indice varia tra -1 e 1: il valore è basso se c'è mancanza di umidità, mentre è elevato se la vegetazione è rigogliosa e ricca d'acqua.\\
Il parametro più importante da ottenere mediante l'indice NDWI8A è la sua variazione nel tempo, in modo da conoscere se ci siano stati o meno cambiamenti delle condizioni al suolo; pertanto è necessario svolgere la formula:
\begin{equation}
    NDWI8A_{diff} = NDWI8A_{pre} - NDWI8A_{post}
\end{equation}
Risultati elevati indicano un'alta variazione delle condizioni tra due momenti temporali diversi.\\
Le maggiori limitazioni sull'utilizzo dell'indice NDWI8A, come per resto anche degli altri, per il rilevamento degli schianti da vento sono:
\begin{itemize}
    \item la risoluzione spaziale limitata, in quanto si prevede l'utilizzo delle sole bande con risoluzione 20 metri;
    \item la dipendenza dalle condizioni atmosferiche e temporali, come per esempio la presenza di neve, ombre e nuvole;
    \item la difficoltà nel rilevare danni diffusi o sparsi, in quanto la bassa risoluzione spaziale performa correttamente analizzando danni intensi, estesi ed uniformi;
    \item le limitazioni altitudinali e stagionali, in quanto la presenza della neve al suolo condiziona in modo cruciale il rilevamento dell'umidità;
    \item gli effetti della fenologia e delle attività umane, in quanto inducono cambiamenti non valutabili da immagini satellitari (come per esempio la rimozione delle piante).
\end{itemize}
Per questa relazione sono stati utilizzati i seguenti prodotti satellitari:
% Please add the following required packages to your document preamble:
% \usepackage{multirow}
\begin{table}[H] \centering
\begin{tabular}{ccccc}
    \toprule
\multicolumn{1}{l}{Satellite} & Giorno & \multicolumn{1}{l}{Bande}  & \multicolumn{1}{l}{Ris. spaz.} &  \\
    \midrule
\multirow{3}{*}{Sentinel-2} & 27/08/2018 & \multirow{3}{*}{\begin{tabular}[c]{@{}c@{}}B8A\\ B11\end{tabular}} & \multirow{3}{*}{20 m}&  \\
         &   23/07/2019     &                  &                  &  \\
        &  12/08/2025      &                   &              &        \\
    \bottomrule
\end{tabular}
\end{table}
Sono stati scelti periodi stagionali simili, pur se per anni diversi, in modo da ridurre l'effetto dell'insolazione differente nei diversi scenari.