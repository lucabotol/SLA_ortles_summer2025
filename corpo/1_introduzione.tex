\section{Introduzione}
In questa relazione si andrà ad esporre il procedimento ed i risultati relativi allo studio della componente di copertura nevosa nel bacino idrografico della Val de la Mare (Provincia Autonoma di Trento), durante il periodo estivo dell'anno 2025.\\
Le informazioni territoriali utilizzate per questo lavoro sono state ricavate maggiormente da rilievi dei satelliti Sentinel-2. \\
Successivamente alla descrizione dell'andamento stagionale delle nevi, ed un inquadramento geografico del sito di studio, si andrà ad esporre i materiali ed i metodi utilizzati.\\
Lo studio della neve in ambiente alpino è molto importante, in quanto ricopre un ruolo importante nel ciclo idrogeologico.\\
Inoltre, l'analisi dell'andamento (stagionale o pluriannuale) della copertura nevosa permette di ottenere informazioni circa il fenomeno di riscaldamento globale \cite{Nagajothi_2021}.\\
Più precisamente, questo lavoro si è focalizzato sul Transient Snow Line (TSL), ovvero la "linea della neve transitoria", del periodo estivo. Questo termine indica il limite visibile che separa la superficie coperta di neve fresca (dell'inverno precedente) dal ghiaccio, dal firn o dal terreno nudo in un dato momento temporale.\\
L'altitudine della TSL varia stagionalmente in risposta alle condizioni climatiche: le precipitazioni nevose invernali ne determinano l'abbassamento, mentre la fusione nivale primaverile ed estiva ne causa la progressiva risalita verso quote più elevate.
