\section{Ecologica dei disturbi}
Da definizione, un disturbo ecologico è ``un qualsiasi evento discretion nel tempo che altera la struttura di un ecosistema, comunità o popolazione e modifica le risorse, le disponibilità di un substrato o dell'ambiente fisico".\\
I disturbi possiedono tre proprietà, ovvero sono:
\begin{itemize}
    \item ubiquitari: ``presenti in ogni luogo";
    \item intrisechi: ``proprio dell'essenza di una cosa";
    \item inevitabili: ``ciò che non si può evitare".
\end{itemize}
Queste proprietà indicano come i disturbi naturali siano parte di ogni sistema ambientale (entro un certo limit sono fisicamente contenibili).\\
Lo studio dei disturbi in ecological è un concetto relativamente recente. Infatti, oggigiorno viene considerata la loro importanza nei sistemi naturali, in quanto sono motori di diversità biologica.\\
Essendo eventi discreti, occorre definire la scala spaziale e temporale dei disturbi. Dal punto di vista forestale, si svolge l'inquadramento spaziale a scala di singoli individui o di popolamenti; per quanto riguarda l'inquadramento temporale, si considera il ciclo vitale delle specie arboree (1-300 anni).\\
I disturbi possono essere categorizzati in base alle loro diverse caratteristiche:
\begin{itemize}
    \item periodo in cui si manifesto (stagionalità);
    \item frequenza (o l'inverso, cioè il tempo di ritorno);
    \item intensità;
    \item dimensione (variazione spaziale);
    \item eterogeneità (variazione d'intensità e severità all'interno dell'area interessata);
    \item durata dell'evento;
    \item biological legacies (quantità e qualità dei residui degli organismi che sopravvivono al disturbo).
\end{itemize}
L'intensità e la severità sono due concetti diversi per quanto riguarda l'ecologia dei disturbi: la prima indica la caratteristica fisica dell'evento (come per esempio la velocità del vento o il volume di massi in movimento), mentre la seconda indica l'effettiva perdita di biomassa del popolamento (per esempio lo schianto di un certo numero di alberi o la perdita di volume di chioma).\\
L'intensità e la severità definiscono assieme la magnitudo di un disturbo; conoscendo la magnitudo è possibile definire la frequenza di accadimento (essendoci la relazione inversamente proporzionale tra di loro).
\subsection{Schianti da vento}
Il vento è una massa d'aria che si sposta, con componente prettamente orizzontale, da zone ad alta pressione a zone a bassa pressione.\\
La velocità del vento (che è il fattore più semplice da misurare) è proporzionale al gradiente barico. La pressione esercitata dal vento aumenta proporzionalmente con il quadrato della sua velocità.\\
Per quanto riguarda i disturbi generati dall'azione del vento, è di maggior interesse la valutazione del carico (load); infatti, mentre la velocità rimane relativamente costante (a parità di altimetria), il carico varia tra pianta a pianta.\\
Oltre a generare disturbi, il vento può essere identificato come un agente esterno di stress (per esempio modificando la forma della chioma degli alberi) ed un fattore limitante alla crescita, ma anche un agente chiave nella riproduzione (disperdendo il polline o disseminando i semi).\\
La velocità del vento non è costante, ma varia in base all'altezza dal suolo, aumentando in modo logistico. Questa caratteristica è importante per quanto riguarda i popolamenti forestali, in quanto la differenza di velocità tra il colletto e la cima è notevole. In base alle diverse velocità, i popolamenti subiranno diverse severità.\\
Questa tipologia di disturbo risulta essere quella più impattante nel contesto forestale europeo. A livello nazionale, gli eventi di questo tipo avvengono maggiormente nel periodo tra ottobre e gennaio.\\
Gli schianti da vento si suddividono in due tipi, in base alla tipologia di danno prodotto:
\begin{itemize}
    \item stroncamento: avviene la rottura del fusto al di sopra del colletto. Se la rottura avviene nella metà superiore del fusto, prende il nome di svettamento;
    \item sradicamento (o ribaltamento): avviene il cedimento dell'ancoraggio radicale, che causa l'abbattimento del fusto e l'estrusione delle radici dal suolo.
\end{itemize}
A scala del popolamento, gli schianti da vento producono l'apertura di piccole buche. Il susseguirsi di schianti da vento apre buche sempre di maggiori dimensioni: un primo disturbo genera margini più fragili rispetto alla condizione pre-evento.\\
Uno schianto da vento si genera dalla combinazione della componente orizzontale del vento e della gravità.\\
I fattori influenti sul disturbo sono:
\begin{itemize}
    \item la densità e velocità dell'aria;
    \item le caratteristiche della chioma (forma, massa, densità,...);
    \item le caratteristiche dell'albero (altezza, diametro del fusto, resistenza a flessione del fusto,...).
\end{itemize}
Il carico del vento agente sulla chioma e sul tronco può essere calcolato mediante la formula:
\begin{equation}
    F_i = \frac{1}{2} \cdot \rho \cdot A_i \cdot C_{Di} \cdot v_i^2
\end{equation}
Dove:
\begin{itemize}
    \item $\rho$ indica la densità dell'aria; 
    \item $A_i$ è l'area d'impatto del vento;
    \item $C_{Di}$ è un correttore, con un range da 0 a 1.
\end{itemize}
La forza del vento viene trasferita al colletto, generando così un momento torcente:
\begin{equation}
    T = \Sigma (F_i \cdot h_i)
\end{equation}
Oltre alla forza del vento, la gravità contribuisce generando un momento flettente sul fusto, considerando l'asse di simmetria della pianta (e di conseguenza la posizione del baricentro):
\begin{equation}
    F_i = m_i \cdot x_i \cdot g
\end{equation}
Dove $x_i$ indica la posizione della massa unitaria rispetto all'asse di simmetria.\\
A scala di popolamento, la resistenza aumenta al diminuire della densità delle piante, soprattutto se la foresta è disetanea.\\
Dal punto di vista selvicolturale, occorre:
\begin{itemize}
    \item ottenere un popolamento disetaneo;
    \item avere un margine permeabile, in modo da ridurre le turbolenze e contemporaneamente anche la velocità del vento;
    \item ridurre l'altezza di alcune piante, o ridurre il volume di chioma.
\end{itemize}
Lo spazio massimo di apertura delle buche dev'essere 2/2.5 volte l'altezza degli alberi di margine.\\
La necromassa generata dagli schianti da vento (sia questa composta da snags, logs o ceppaie), ricopre un ruolo cruciale nel processo di rinnovazione: infatti, oltre ad essere una fonte di sostanza organica per le nuove piantine, svolge il ruolo di protezione dal brucamento animale o dallo scivolamento nivale. 
\subsection{La tempesta Vaia}
La tempesta Vaia è stata una forte perturbazione accaduta tra il 26 ottobre ed il 5 novembre del 2018 \cite{wikipedia_vaia}.\\
La stima definitiva dei danni prodotti da Vaia si attesta a 16,5 milioni di metri cubi di biomassa schiantata (di cui 3 milioni in Veneto).\\
I danni prodotti sono stati così elevati principalmente per due motivi:
\begin{itemize}
    \item una serie di precipitazioni intense, soprattutto nei giorni tra il 27 ed il 28 ottobre (media di 335 mm);
    \item un'azione del vento notevole, il massimo è stato registrato nel Monte Cesen il 29 ottobre (192.5$km/h$).
\end{itemize}
La prima fase ha generato principalmente eventi franosi, lave torrentizie ed inondazioni. La seconda fase, causa l'azione del vento, ha generato schianti da vento \cite{vaiaLandStoryMap}.\\
A seguito della prima fase emergenziale, sono state effettuate una serie di interventi selvicolturali di utilizzazione forestale, ovvero la ``timber extraction'' ed il ``salvage logging''.\\
Il periodo successivo all'evento Vaia è stato interessato dalla pullulazione del bostrico (\textit{Ips typographus}), portando alla morte un volume di biomassa superiore a quello generato dalla tempesta. 