\section{Discussioni dei risultati e confronto con un altro caso studio}
\subsection{Risultati raccolti}
Analizzando i risultati ottenuti dalla raccolta dei dati in campo, è possibile notare come, le condizioni di copertura del suolo siano molto diverse.\\
Per quanto riguarda la prima area di studio (n.2), presenta la maggior quantità di necromassa e rinnovazione arborea rispetto alle altre superfici di saggio.\\
L'area di studio 6 possiede invece la minore quantità di necromassa a terra e la minore copertura di rinnovazione arborea.\\
Per esempio, la notevole differenza di copertura arborea (rinnovazione) e necromassa (protezione e fonte di risorse) tra il sito 2 ed il sito 6 suggeriscono quale sarà la velocità di ripristino del soprassuolo maggiore, poichè teoricamente la più favorevole.
\subsection{Caso studio}
Al fine di implementare questa relazione con uno studio relativo allo stesso disturbo osservato in campo, è stato scelto di esporre un articolo inerente ai fattori che guidano la rinnovazione post-evento.\\
L'articolo in questione, del 2014 e con autore principale Kramer, è intitolato: ``\textit{Site factors are more important than salvage logging for tree regeneration after wind disturbance in Central European forests}" \cite{KRAMER2014}.\\
Analizzando 89 siti svizzeri soggetti a schianti da vento, causati dalle tempeste Vivian (1990) e Lothar (1999), i ricercatori sono andati a studiare i processi di rinnovazione in atto, focalizzandosi su diverse variabili:
\begin{itemize}
    \item elevazione (m s.l.m.);
    \item esposizione;
    \item profondità e pH del suolo;
    \item area della buca;
    \item pendenza;
    \item trattamento post-evento;
    \item copertura della vegetazione erbacea;
    \item distanza temporale dal disturbo.
\end{itemize}
I siti presi come studio si posizionano tra i 695 ed i 1429 m s.l.m.\\
In particolare, è stato fatto un focus sul trattamento post-evento del salvage-logging. Questa operazione viene svolta spesso principalmente per motivi economici, dato il gran volume di assortimenti legnosi che possono essere portati fuori foresta.\\
D'altro canto, data la quasi completa rimozione del soprassuolo forestale, il salvage logging riduce l'eterogeneità strutturale, con notevoli impatti per gli habitat ed i microhabitat.\\
D'altro canto, il salvage logging si può rivelare utile in quanto riduce i tempi naturali della rinnovazione.\\
L'effetto positivo della necromassa, per quanto riguarda l'influenza sulla lettiera, si manifesta successivamente ad una certa decomposizione del materiale biologico. Tale periodo si attesta a circa 10-20 anni; pertanto, solitamente non si riescono a notare benefici da parte della necromassa sul suolo nel periodo immediatamente successivo al disturbo.\\
In conclusione, l'articolo riporta che la rinnovazione 
(a distanza di 10-20 anni dal disturbo), risultava maggiore nei siti:
\begin{itemize}
    \item con suoli a maggior valore di pH;
    \item a ridotta competizione intraspecifica;
    \item a bassa quota;
    \item dove è stato svolto il salvage logging.
\end{itemize}
Come descritto nella sezione precedente è possibile valutare, allo stato attuale, le condizioni di rinnovazione dei siti, avendo solamente le conoscenze sulla necromassa e sulla rinnovazione. Volendo ordinare, secondo lo stato qualitativo della rinnovazione, il sito n.2 risulta il migliore, seguito dal n.4 e dal n.6.
%dalle info dell'articolo, fare un commento sui risultati%